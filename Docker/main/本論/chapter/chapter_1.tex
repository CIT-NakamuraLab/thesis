\chapter{Texの書き方(基本編)}
\label {chp:tex_basic}

\section{章,節の書き方}
\label{sec:tex_basic_section}
\subsection{基本}
\label{sub:tex_basic_section_basic}
    文章を書く際に使用する書式です.\\
    章節は目次に表示されます.\\
    |記述例|
    \begin{verbatim}
    \chapter{章の名前}
    \section{節の名前}
    \subsection{小節の名前}
    \end{verbatim}

\section{改行,コメントアウト}
\label{sec:tex_basic_newline}
    改行は\verb|\\|というように,バックスラッシュ(円マーク)を続けて二つ記述します.\\
    \%を書くと以降がコメントアウトされます.

    エディタ上で一行空白をあけて記述することで,新たな文節として認識されます.

\section{環境とコマンド}
\label{sec:tex_basic_envcmd}
    Latexには「環境」と「コマンド」があります.
    環境は複数行にまたがるもので,コマンドは一行のみ有効なものです.
\subsection{環境}
\label{sub:tex_basic_section_envcmd_env}
    環境を使用するときは\verb|\begin{環境名}と\end{環境名}|で囲みます.\\
    画像を表示,表を作成,箇条書き等さまざまな場面で使用します.
\subsection{コマンド}
\label{sub:tex_basic_section_envcmd_cmd}
    コマンドは\verb|\newpage|のように先頭に\verb|\|をつけた単語を記述します.\\
    環境と同様にさまざまな種類があります.

\section{エスケープ}
\label{sec:tex_basic_escape}
    \# \$ \& \% \{ \}  \\
    などがエスケープの必要な文字です.直前に\verb|\|をつけるとエスケープされます.\\
    文章全体をエスケープする場合は「verbatim」環境を使用します.
     
\section{箇条書き}
\label{sec:tex_basic_clause}
    箇条書きを行う場合にはitemize環境を使用します.\\
    改行を入れることで題目と説明のように表示することが可能です.\\
    |記述例|\\
    \verb|\begin{itemize}|\\
        \verb|\item| あいてむ1\\
        \verb|\item| あいてむ2\\\\
        あいてむ2について\\
    \verb|\end{itemize}|\\\\
    |表示例|
\begin{itemize}
    \item あいてむ1
    \item あいてむ2

    あいてむ2についてあいてむ2についてあいてむ2についてあいてむ2についてあいてむ2についてあいてむ2についてあいてむ2についてあいてむ2についてあいてむ2についてあいてむ2についてあいてむ2についてあいてむ2についてあいてむ2についてあいてむ2についてあいてむ2について
\end{itemize} 

    箇条書きには複数種類があります.\\
    itemize環境の場合は通常の「・」,enumerate環境の場合は「1.」のように数字列挙に,description環境の場合は「\verb|\item[項目A] 説明文|」と書くことで項目つきの箇条になります.